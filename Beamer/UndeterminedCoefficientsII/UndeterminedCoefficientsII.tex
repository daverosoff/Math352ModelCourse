\def\encoding{UTF-8}
\input{mmd-beamer-header-rosoff}
\def\mytitle{Undetermined Coefficients, II}
\def\subtitle{Products and degeneracy}
\def\affiliation{The College of Idaho}
\def\myauthor{Math 352 Differential Equations}
\def\mydate{8 April 2013}
\def\latexmode{beamer}
\def\fonttheme{structurebold}
\def\colortheme{crane}
\def\theme{Szeged}
\input{mmd-beamer-begin-doc-rosoff}
\def\htmlheaderlevel{2}
\section{Recap}
\label{recap}

\begin{frame}

\frametitle{The nondegenerate case}
\label{thenondegeneratecase}

Last week, we saw how to use the 
\textsc{Method of Undetermined Coefficients}
 to find particular solutions of the inhomogeneous equation
\[
    ay'' + by' + cy = g(t),
\]
when $ g(t) $ is a polynomial, an exponential function, or a linear combination of sines and cosines (with like frequencies).

Our findings are summarized in the next table, and should be memorized.

\end{frame}

\begin{frame}

\frametitle{The shape of $Y$ for simple $ g $}
\label{theshapeofyforsimpleg}

\begin{table}[htbp]
\begin{minipage}{\linewidth}
\setlength{\tymax}{0.5\linewidth}
\centering
\small
\begin{tabulary}{\textwidth}{@{}CC@{}} \toprule
$ g(t) $&$ Y(t) $\\
\midrule
$ t^n $&$ A_n t^n + \cdots + A_ 0 $\\
$ \exp{(at)} $&$ A \exp{(at)} $\\
$ a \cos{(\omega t)} $&$ A \cos{(\omega t)} + B \sin{(\omega t)} $\\
$ a \sin{(\omega t)} $&$ A \cos{(\omega t)} + B \sin{(\omega t)} $\\

\bottomrule

\end{tabulary}
\end{minipage}
\end{table}


We saw too that if $ g(t) $ is a linear combination of the above forms, then an appropriate guess for $ Y $ is a linear combination (with suitably generalized coefficients) of the corresponding entries in the second column. For example, $ g(t) = 3t^2 + \exp{(2t)} $ would lead to the form
\[
    Y(t) = At^2 + Bt + C + D \exp{(2t)}.
\]

\end{frame}

\section{Adaptations, variants, and degeneracy}
\label{adaptationsvariantsanddegeneracy}

\begin{frame}

\frametitle{Efficiency in choosing the form of $ Y $}
\label{efficiencyinchoosingtheformofy}

The table is given in (almost) the most abbreviated form. That means there is a lot of opportunity to waste time, if you are not paying attention to what you're doing. Suppose $ g(t) = 3t^3 + t^4 $. Each of these is a polynomial, so a suitable choice for $ Y(t) $ could be
\[
    Y(t) = \underbrace{At^3 + Bt^2 + Ct + D}_{\text{from $3t^3$}} + \underbrace{Et^4 + Ft^3 + Gt^2 + Ht + I}_{\text{from $t^4$}}.
\]
The problem here is that this is very inefficient. It is a large system of equations awaiting you, with many solutions (rather than just one). 

\end{frame}

\begin{frame}

\frametitle{How to be efficient}
\label{howtobeefficient}

The reason is that this polynomial is a disguised version of
\[
    Y(t) = At^4 + Bt^3 + Ct^2 + Dt + E
\]
which is what you get if you regard $ g(t) $ as a single polynomial, rather than as a linear combination. Always think of $ g(t) $ in the way that leads to the fewest terms in the linear combination.

\end{frame}

\begin{frame}

\frametitle{Products}
\label{products}

There are similar improvements to be made when confronted with an inhomogeneous term that is a product of the atomic ones, for example
\[
    g(t) = \exp{(2t)} \sin{(t)}.
\]
Now the na\"ive guess is $ Y = A \exp{(2t)} \sin{(t)} $, but having been burned before we know better and write down the product of \emph{the corresponding table entries},
\[
    Y(t) = (A \exp{(2t)}) (B \cos{(t)} + C \sin{(t)}).
\]
This looks more promising, and an expansion gives
\[
    Y(t) = AB \exp{(2t)} \cos{(t)} + AC \exp{(2t)} \sin{(t)}.
\]

\end{frame}

\begin{frame}

\frametitle{Relabel or reorganize}
\label{relabelorreorganize}

It is essential to remember that only the \emph{form} of $ Y $ matters, not the names we give to its coefficients. Thus $ AB $ and $ AC $ should just be relabeled. It's OK to reuse the letters, so you could write
\[
    Y(t) = A \exp{(2t)} \sin{(t)} + B \exp{(2t)} \cos{(t)}.
\]
This is more efficient because it is a \emph{smaller} system with \emph{fewer} solutions.

Another conceptual approach is to think about exponential factors separately. Observe that $ Y $ in the previous display is the product of the \emph{original} exponential from $ g $ with the \emph{prescribed guess} for the trignonometric factor. This ``works'' with polynomials also. 

\end{frame}

\begin{frame}

\frametitle{Try these out}
\label{trytheseout}

Write down the most efficient form for $ Y $ that you can, for each of these instances of $ g(t) $.


\[
    e^{7t} (2 \cos{(2t)} - 8 \sin{(2t)}) \only<2>{ \implies A e^{7t} \cos{(2t)} + B e^{7t} \sin{(2t)} }
\] \[
    t^2 e^{2t} \only<2>{ \implies A t^2 e^{2t} + B t e^{2t} + C e^{2t}} 
\] \[
    3t^3 \sin{5t} \only<2>{\quad \text{...shown on the next slide.}}
\]


\end{frame}

\begin{frame}

\frametitle{Not too cool}
\label{nottoocool}


\begin{align*}
At^3 \cos{(5t)} &+ Bt^3 \sin{(5t)} + Ct^2 \cos{(5t)} + Dt^2 \sin{(5t)} \\ 
&+ Et \cos{(5t)} + Ft \sin{(5t)} + G \cos{(5t)} + H \sin{(5t)}
\end{align*}


This results in an $ 8 \times 8 $ system of equations, which I wouldn't ask you to solve by hand. We'll see how to solve them using Sage in a couple of weeks.

\end{frame}

\begin{frame}

\frametitle{Degeneracy}
\label{degeneracy}

One question remains that we haven't gone into: what to do when part of the complementary solution $ c_1 y_1 + c_2 y_2 $ appears in $ g(t) $?
\[
    y'' + 5y' + 6y = 6e^{-2t}
\]
No choice of $ A $ will make $ Y = A \exp{(-2t)} $ a solution to the above inhomogeneous equation, because such a function is a solution to the \emph{associated homogeneous equation}. In such cases, guided either by past experience or divine inspiration, one uses a higher-degree polynomial. Let the coefficient be $ At $ rather than $ A $.

\end{frame}

\begin{frame}

\frametitle{A degenerate solution}
\label{adegeneratesolution}

Here, you can check that $ 6t \exp{(-2t)} $ is a solution. It is found by the same method as usual. But wait, you say!
 \pause 
What if we consider the equation $ y'' + 4y' + 4y = 6 \exp{(-2t)} $?
 \pause 
If a little medicine is good, then a lot of medicine must be even better. So use a second-degree polynomial, and try to find $ Y $ of the form
\[
    Y(t) = At^2 e^{-2t}.    
\]

\end{frame}

\begin{frame}

\frametitle{The use of the method}
\label{theuseofthemethod}

For even mildly complicated functions $ g(t) $ the equations that result quickly become intractable for hand calculation. Yet knowing the technique of guessing the form of $ Y $ is useful, because a computer algebra system can be used to do the rest of the work of searching for the coefficients.

Once again I urge you to resist the impulse to memorize the table in the text with information about the degeneracies; just remember to increase the degree of the coefficient. It's ok to memorize the table that appears in this presentation, though.

\end{frame}

\begin{frame}

\frametitle{Coming attractions}
\label{comingattractions}

Undetermined coefficients doesn't work as nicely if the inhomogeneous term $ g(t) $ isn't a linear combination of products of polynomials, trig functions, and exponentials. For that reason, we must investigate a second method: \textsc{variation of parameters.}

\end{frame}

\mode<all>
\input{mmd-beamer-footer}

\end{document}\mode*

