\documentclass{beamer}

\mode<presentation>
{
  \usetheme{Szeged}      % or try Darmstadt, Madrid, Warsaw, ...
  \usecolortheme{crane} % or try albatross, beaver, crane, ...
  \usefonttheme{structurebold}  % or try serif, structurebold, ...
  \setbeamertemplate{navigation symbols}{}
  \setbeamertemplate{caption}[numbered]
} 

\usepackage{paralist}
\usepackage{xparse}
\usepackage[english]{babel}
\usepackage[utf8x]{inputenc}

\NewDocumentCommand\C{}{\mathbf{C}}

% On writeLaTeX, these lines give you sharper preview images.
% You might want to comment them out before you export, though.
%\usepackage{pgfpages}
%\pgfpagesuselayout{resize to}[%
%  physical paper width=8in, physical paper height=6in]

\title[Complex Numbers]{A brief introduction to complex numbers \\
for solving linear equations}
\author{D.\ Rosoff}
\institute{College of Idaho}
\date{15 March 2013}

\begin{document}

\begin{frame}
  \titlepage
\end{frame}

% Uncomment these lines for an automatically generated outline.
%\begin{frame}{Outline}
%  \tableofcontents
%\end{frame}

\section{Recap}

\begin{frame}{Last time}

This week, we've been investigating the second-order homogeneous equation
\[
   ay'' + by' + cy = 0.
\]

\begin{itemize}
  \item Monday, we saw how to find solutions of exponential type $y = ce^{rt}$. The growth constants $r$ that ``work'' are the roots of the characteristic equation $ar^2 + br + c = 0$.
  \pause
  \item When $b^2 - 4ac > 0$ (the characteristic polynomial has \emph{positive discriminant}), we get two values $r_1$ and $r_2$. 
  \pause
  \item Hence we obtain two solutions, $y_1 = e^{r_1 t}$ and $y_2 = e^{r_2 t}$.
\end{itemize}

\end{frame}

\begin{frame}{Recap}

We also know that solutions of \emph{linear homogeneous equations} can be linearly combined to get more solutions. Therefore, we have a 2-dimensional family of solutions
\[
  c_1 y_1 + c_2 y_2.
\]

\begin{block}{Exercise}
Use the Wronskian to show that when $b^2 - 4ac > 0$, the solutions $y_1$ and $y_2$ described above generate all solutions to $ay'' + by' + cy = 0$.
\end{block}

\end{frame}

\begin{frame}{Negative discriminant}

  None of this tells what happens when $b^2 - 4ac < 0$. The existence theorem still applies, so the solutions have to be out there somewhere. If we expand our thinking to the realm of \emph{complex numbers}, we can get them using the same trick.

  \begin{block}{Definition}
    A \emph{complex number} is a symbol $x + iy$, where $x$ and $y$ are ordinary (i.e., real) numbers. We adopt two conventions:
    \begin{inparaenum}[(1)]
      \item these numbers obey all the rules of arithmetic;
      \item $i^2 + 1 = 0$.
    \end{inparaenum}
  \end{block}

\end{frame}

\begin{frame}{Why bother?}

  Our polynomial trick relies on getting the roots of $ar^2 + br + c = 0$. They only exist in the extended algebraic system of complex numbers. You are probably aware that polynomials such as $x^2 + 9 = 0$ admit solutions in the complex numbers. There is a somewhat more general statement to make.

  \begin{block}{Theorem}
    (Fundamental Theorem of Algebra.) Each polynomial of degree $n$ with complex coefficients has exactly $n$ complex roots, counting multiplicities. Equivalently, each such polynomial splits over the complex numbers into $n$ (not necessarily distinct) linear factors.
  \end{block}

  This theorem was known to Euler, but it is widely asserted that the first satisfactory proofs were due to Gauss (1777--1855). 

\end{frame}

\section{Complex polynomials}

\begin{frame}{Getting solutions}
  
  The quadratic formula works in complete generality in the complex context. We will always assume that $a$, $b$, and $c$ (the coefficients in our differential equation) are real. This avoids some interesting difficulties with the complex square root. Observe that if $D < 0$, then $D = -|D| = i^2 |D|$.

  \begin{block}{Square roots of negative numbers}
  We agree to interpret (when $D = b^2 - 4ac < 0$) the symbol $\sqrt{D}$ as:
    \[
      \sqrt{D} = \sqrt{i^2 |D|} = i \sqrt{|D|}.
    \]
  \end{block}

\end{frame}

\section{The complex exponential function}

\begin{frame}{The complex exponential}

  Once we have obtained our two complex roots $r_1$ and $r_2$, we need to interpret the corresponding exponential solutions. 
  \pause
  \begin{block}{WTF?}
    What's $e^{rt}$ when $r$ is complex?
  \end{block}
  
  The answer is found in what is called Euler's formula.
\end{frame}

\begin{frame}{Euler's formula}
  
  A historical quotation on Euler's formula:

  \begin{block}{Benjamin Peirce (1809--1880)}
    ``Gentlemen, that is surely true,
it is absolutely paradoxical;
we cannot understand it,
and we don't know what it means.
But we have proved it,
and therefore we know it must be the truth.''
  \end{block}

\end{frame}

\begin{frame}{Euler's formula}
  
  Since $r$ is complex, it is a symbol of the form $x + iy$. Therefore, since $t$ is real, we get

  \[
    e^{rt} = e^{(x+iy)t} = e^{xt} e^{iyt}.
  \]

  This is assuming the addition law for a function we haven't really defined, but if it didn't obey the addition law, we couldn't bear to call it the exponential function.
\end{frame}

\begin{frame}{Euler's formula}
  
  This famous and celebrated equation tells us how to interpret the $e^{iyt}$ portion of our exponential expression.

  \begin{block}{Euler's formula}
    \[e^{i \theta} = \cos{\theta} + i \sin{\theta}\]
  \end{block}
 
  Thus $\exp(rt) = \exp(xt + iyt) = e^{xt}(\cos{(yt)} + i \sin{(yt)})$.

\end{frame}

\begin{frame}{Euler's formula}
  Putting $\theta = \pi$, we obtain a special case, often referred to as Euler's formula.

  \begin{block}{Also Euler's formula}
    \[ e^{\pi i} = -1. \]
  \end{block}
  
  Respect.

\end{frame}

\end{document}
