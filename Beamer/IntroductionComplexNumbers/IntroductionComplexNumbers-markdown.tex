\def\encoding{UTF-8}
\input{mmd-beamer-header}
\def\mytitle{Complex Numbers}
\def\subtitle{A brief introduction to complex numbers for solving linear equations}
\def\myauthor{D. Rosoff}
\def\affiliation{The College of Idaho}
\def\mydate{15 March 2013}
\def\latexmode{beamer}
\def\fonttheme{structurebold}
\def\colortheme{crane}
\def\theme{Szeged}
\input{mmd-beamer-begin-doc-rosoff}
 \usepackage{inparaenum} 

\section{Recap}
\label{recap}

\begin{frame}

\frametitle{Last time: the exponential trick}
\label{lasttime:theexponentialtrick}

This week, we've been investigating the second-order homogeneous equation
\[
   ay'' + by' + cy = 0.
\]

\begin{itemize}
\item Monday, we saw how to find solutions of exponential type $y = ce^{rt}$. The growth constants $r$ that ``work'' are the roots of the characteristic equation $ar^2 + br + c = 0$.

\item When $b^2 - 4ac > 0$ (the characteristic polynomial has \emph{positive discriminant}), we get two values $r_1$ and $r_2$.

\item Hence we obtain two solutions, $y_1 = e^{r_1 t}$ and $y_2 = e^{r_2 t}$.

\end{itemize}

\end{frame}

\begin{frame}

\frametitle{Last time: superposition}
\label{lasttime:superposition}

We also know that solutions of \emph{linear homogeneous equations} can be linearly combined to get more solutions. Therefore, we have a 2-dimensional family of solutions
\[
  c_1 y_1 + c_2 y_2.
\]

 \begin{block}{Exercise} 

Use the Wronskian to show that when $b^2 - 4ac > 0$, the solutions $y_1$ and $y_2$ described above generate all solutions to $ay'' + by' + cy = 0$.
 \end{block} 

\end{frame}

\begin{frame}

\frametitle{Negative discriminant}
\label{negativediscriminant}

None of this tells what happens when $b^2 - 4ac < 0$. The existence theorem still applies, so the solutions have to be out there somewhere. If we expand our thinking to the realm of \emph{complex numbers}, we can get them using the same trick.

 \begin{block}{Definition} 
 A \emph{complex number} is a symbol $x + iy$, where $x$ and $y$ are ordinary (i.e., real) numbers. We adopt two conventions:
  \begin{inparaenum}[(1)] 
  \item  these numbers obey all the rules of arithmetic;
  \item  $i^2 + 1 = 0$.
  \end{inparaenum} 
  \end{block} 

\end{frame}

\begin{frame}

\frametitle{Why bother?}
\label{whybother}

Our polynomial trick relies on getting the roots of $ar^2 + br + c = 0$. They only exist in the extended algebraic system of complex numbers. You are probably aware that polynomials such as $x^2 + 9 = 0$ admit solutions in the complex numbers. There is a somewhat more general statement to make.

 \begin{block}{Theorem} 
 (Fundamental Theorem of Algebra.) Each polynomial of degree $n$ with complex coefficients has exactly $n$ complex roots, counting multiplicities. Equivalently, each such polynomial splits over the complex numbers into $n$ (not necessarily distinct) linear factors.
  \end{block} 

This theorem was known to Euler, but it is widely asserted that the first satisfactory proofs were due to Gauss (1777--1855). 

\end{frame}

\section{Complex polynomials}
\label{complexpolynomials}

\begin{frame}

\frametitle{Getting solutions}
\label{gettingsolutions}

The quadratic formula works in complete generality in the complex context. We will always assume that $a$, $b$, and $c$ (the coefficients in our differential equation) are real. This avoids some interesting difficulties with the complex square root. Observe that if $D < 0$, then $D = -|D| = i^2 |D|$.

 \begin{block}{Square roots of negative numbers} 
 We agree to interpret (when $D = b^2 - 4ac < 0$) the symbol $\sqrt{D}$ as:
 \[
      \sqrt{D} = \sqrt{i^2 |D|} = i \sqrt{|D|}.
    \]
  \end{block} 

\end{frame}

\section{The complex exponential function}
\label{thecomplexexponentialfunction}

\begin{frame}

\frametitle{The complex exponential}
\label{thecomplexexponential}

Once we have obtained our two complex roots $r_1$ and $r_2$, we need to interpret the corresponding exponential solutions.
  \pause 
  \begin{block}{WTF?} 
 What's $e^{rt}$ when $r$ is complex?
  \end{block} 

The answer is found in what is called Euler's formula.

\end{frame}

\begin{frame}

\frametitle{Euler's formula}
\label{eulersformula}

A historical quotation on Euler's formula:

 \begin{block}{Benjamin Peirce (1809--1880)} 

``Gentlemen, that is surely true,
it is absolutely paradoxical;
we cannot understand it,
and we don't know what it means.
But we have proved it,
and therefore we know it must be the truth.''
 \end{block} 

\end{frame}

\begin{frame}

\frametitle{Euler's formula}
\label{eulersformula}

Since $r$ is complex, it is a symbol of the form $x + iy$. Therefore, since $t$ is real, we get

\[
    e^{rt} = e^{(x+iy)t} = e^{xt} e^{iyt}.
  \]

This is assuming the addition law for a function we haven't really defined, but if it didn't obey the addition law, we couldn't bear to call it the exponential function.

\end{frame}

\begin{frame}

\frametitle{Euler's formula}
\label{eulersformula}

This famous and celebrated equation tells us how to interpret the $e^{iyt}$ portion of our exponential expression.

 \begin{block}{Euler's formula} 
 \[e^{i \theta} = \cos{\theta} + i \sin{\theta}\]
  \end{block}  

Thus $\exp(rt) = \exp(xt + iyt) = e^{xt}(\cos{(yt)} + i \sin{(yt)})$.

\end{frame}

\begin{frame}

\frametitle{Euler's formula}
\label{eulersformula}

Putting $\theta = \pi$, we obtain a special case, often referred to as Euler's formula.

 \begin{block}{Also Euler's formula} 
 \[ e^{\pi i} = -1. \]
  \end{block} 

Respect.

\end{frame}

\mode<all>
\input{mmd-beamer-footer}

\end{document}\mode*

