\def\encoding{UTF-8}
\input{mmd-beamer-header-rosoff}
\def\mytitle{Further investigation of harmonic vibration}
\def\affiliation{The College of Idaho}
\def\myauthor{Math 352 Differential Equations}
\def\mydate{17 April 2013}
\def\latexmode{beamer}
\input{mmd-beamer-begin-doc-rosoff}
\def\htmlheaderlevel{2  \\
\# Recap}
\begin{frame}

\frametitle{Last time: free harmonic oscillations}
\label{lasttime:freeharmonicoscillations}

Recall the equation of motion for an unforced spring-mass system:
\[
    mu'' + \gamma u' + k u = 0,
\]
where $ m $, $ k > 0 $ and $ \gamma \geq 0$.

If $ \gamma = 0 $, then our system is a simple harmonic oscillator, vibrating subject to the displacement function
\[
    u = c_1 \cos{(\omega_0 t)} + c_2 \sin{(\omega_0 t)}.
\]
Such a system's motion persists indefinitely. The energy added by the initial conditions stays in the system forever.

\end{frame}

\begin{frame}

\frametitle{Reduction to standard form}
\label{reductiontostandardform}

Every linear combination of sines and cosines with like frequency can be written as a single sinusoidal function. A sinusoidal function is one of the form
$ R \cos{(\omega t - \delta)} $,
where $ R $ is the amplitude, $ \omega $ is the common frequency, and $ \delta $ is the phase shift.

\end{frame}

\begin{frame}

\frametitle{Getting the new parameters}
\label{gettingthenewparameters}

Suppose that we have already obtained $ c_1 $ and $ c_2 $ from the initial conditions and wish to find $ R $ and $ \delta $ with
\[
    c_1 \cos{(\omega_0 t)} + c_2 \sin{(\omega_0 t)} = R \cos{(\omega_0 t - \delta)}.
\]
Using the cosine subtraction identity, we find this entails that
\[
    c_1 \cos{(\omega_0 t)} + c_2 \sin{(\omega_0 t)} = R \cos{\delta} \cos{(\omega_0 t)} + R \sin{\delta} \sin{(\omega_0 t)}.
\]
Hence $ c_1 = R \cos{\delta} $, $ c_2 = R \sin{\delta} $, and the usual polar-coordinate equations give us
\[
    R = \sqrt{c_1^2 + c_2^2}, \tan{\delta} = c_2/c_1.
\]
The arctangent function must be used with due care.

\end{frame}

\begin{frame}

\frametitle{Classification of damping; overdamped}
\label{classificationofdampingoverdamped}

If $ \gamma > 0 $, we refer to the system as ``damped''. The type of damping corresponds to the discriminant $ D = \sqrt{\gamma^2 - 4km} $ of the characteristic polynomial.

When $ D > 0 $, the roots of the characteristic polynomial are real \emph{and negative}. This is the overdamped case, and the displacement function is a linear combination of two exponentials $ e^{r_1 t} $ and $ e^{r_2 t} $. Since $ r_1 $, $ r_2 < 0 $, the vibration decays as $ t $ increases.

\end{frame}

\begin{frame}

\frametitle{Underdamped; critically damped}
\label{underdampedcriticallydamped}

When $ D < 0 $, the roots are complex \emph{with negative real part}, so the oscillation again decays. Write $ \lambda \pm i \mu $ for the roots: then the displacement function is a linear combination of the functions $ e^{\lambda t} \cos{(\mu t)} $ and $ e^{\lambda t} \sin{(\mu t)} $.

\pause

When $ D = 0 $, the system is \emph{critically damped}. Then, there is only one root $ r $ of the characteristic polynomial. The displacement function is a linear combination of $ e^{r t} $ and $ te^{rt} $. The graphs of critically damped displacement functions look a lot like those of overdamped ones.

\end{frame}

\begin{frame}

\frametitle{The damped cases: three regimes}
\label{thedampedcases:threeregimes}

If $ \gamma > 0 $, then the initial energy is eventually (and in practice, quickly) dissipated in resisting the damping force of the surrounding fluid. Clearly, greater values of $ \gamma $ mean ``more'' damping is occurring. The correct way to measure the ``size'' of the damping is not via $ \gamma $ alone, but with a dimensionless coefficient involving all three constants $ m $, $ \gamma $, and $ k $.

\end{frame}

\begin{frame}

\frametitle{Damping and the discriminant}
\label{dampingandthediscriminant}

Let $ Q = \gamma^2 / 4km $. If you compare the dimensions of the three coefficients, you will see that $ Q $ is dimensionless: all the units cancel out of it. Dimensionless coefficients are important, because they don't depend on our scale of measurement. It turns out that $ Q $ is a nice code for the damping type of our system. 

\begin{itemize}
\item Trivially, $ Q = 0 $ when $ \gamma = 0 $.\pause

\item Our system is underdamped when $ 0 < Q < 1 $.\pause

\item Critical damping obtains when $ Q = 1 $.\pause

\item Overdamping is the case $ Q > 1 $.

\end{itemize}

\end{frame}

\begin{frame}

\frametitle{Quasiperiod and quasifrequency}
\label{quasiperiodandquasifrequency}

The parameter $ \mu $ determines the quasifrequency of a damped oscillation (since it is not periodic, it doesn't have an honest ``frequency''). Some algebra shows that
\[
    \frac{\mu}{\omega_0} = \frac{\sqrt{4km - \gamma^2}}{2m \sqrt{k/m}} = \left( 1 - Q \right)^{1/2} \approx 1 - \frac{Q}{2}.
\]
The last approximation is valid, as usual, when $ Q $ is small.

These calculations will be of great utility for us in the next section, which concerns \emph{forced vibrations}.
Thus, small damping slightly reduces the frequency of the oscillation.

\end{frame}

\mode<all>
\input{mmd-beamer-footer}

\end{document}\mode*

