\def\encoding{UTF-8}
\input{mmd-beamer-header-rosoff}
\def\mytitle{Undetermined Coefficients,I}
\def\subtitle{Guess-and-check: expert level}
\def\affiliation{The College of Idaho}
\def\myauthor{Math 352 Differential Equations}
\def\mydate{1 April 2013}
\def\latexmode{beamer}
\def\fonttheme{structurebold}
\def\colortheme{crane}
\def\theme{Szeged}
\input{mmd-beamer-begin-doc-rosoff}
\def\htmlheaderlevel{2}
\section{Recap}
\label{recap}

\begin{frame}

\frametitle{The homogeneous case}
\label{thehomogeneouscase}

In the last couple weeks, we saw how to use the exponential trick and some inspired guessing to get general solutions of the general second-order linear homogeneous equation with constant coefficients,
\[ ay'' + by' + cy = 0. \]

There were three distinct cases, corresponding to the three different possibilities $ D = b^2 - 4ac > 0 $, $ D < 0 $, and $ D = 0 $.

\end{frame}

\section{Inhomogeneous equations}
\label{inhomogeneousequations}

\begin{frame}

\frametitle{Inhomogeneity}
\label{inhomogeneity}

One of the key algebraic coincidences that allowed us to use the exponential trick was the \emph{superposition principle}. This is the principle that if $ y_1 $ and $ y_2 $ are solutions, then all their linear combinations are too. It fails spectacularly if we try it on an inhomogeneous equation, as illustrated in several weekly homework problems.

\end{frame}

\begin{frame}

\frametitle{Inhomogeneity: the translation principle}
\label{inhomogeneity:thetranslationprinciple}

But this doesn't mean that our work on the homogeneous equation won't help us solve equations of the form

\begin{equation} \label{eq:inhom}
ay'' + by' + cy = g(t).
\end{equation}

This is because any two solutions of such an equation differ by a solution of the \emph{associated homogeneous equation} $ ay'' + by' + cy = 0 $. That is to say, if $ Y_1 $ and $ Y_2 $ are both solutions of Equation~\ref{eq:inhom}, then $ Y_1 - Y_2 $ is a solution of the associated homogeneous equation.

\end{frame}

\begin{frame}

\frametitle{The geometry of linear differential operators: homogeneous equations}
\label{thegeometryoflineardifferentialoperators:homogeneousequations}

It is best to think of the whole collection of solutions of the associated homogeneous equation as a set\footnote{Many students have learned to find the word ``set'' unnerving, but it only serves to bind related objects into a conceptual whole. It is a ``one'' that embodies or instantiates a ``many''.} $ S $. A pair of fundamental solutions $ y_1 $, $ y_2 $ matches $ S $ with the Euclidean plane; namely, the point $ (c_1, c_2) $ corresponds to the solution $ c_1 y_1 + c_2 y_2 $. Practice visualizing the plane as embedded in a bigger-dimensional space (the space of all differentiable functions, maybe).

Observe that the point $ (0,0) $ is an element of our plane $ S $, because the zero function is a solution of every homogeneous differential equation. Thus the plane you are imagining passes through the origin of whatever space it lives in.

\end{frame}

\begin{frame}

\frametitle{The geometry of linear differential operators: inhomogeneous equations}
\label{thegeometryoflineardifferentialoperators:inhomogeneousequations}

Since the inhomogeneous equation $ ay'' + by' + cy = g(t) $ is a second-order equation, intuition and the theory of Wronskians tell us that there should be a pair of fundamental solutions. This sort of happens, but the details are a little different.

\begin{block}{The solutions are still a plane}
Solutions of the inhomogeneous equation correspond to points in a plane just like the solutions of the homogeneous equation do. The difference is, it's not the same plane.
\end{block}

The plane for inhomogeneous equations doesn't pass through the origin, because the zero function isn't a solution of any inhomogeneous equation.

\end{frame}

\begin{frame}

\frametitle{The geometry of linear differential operators: particular solutions}
\label{thegeometryoflineardifferentialoperators:particularsolutions}

Recall the translation principle for the inhomogeneous equation:

\begin{block}{Translation principle}
If $Y_1$ and $Y_2$ are solutions of the inhomogeneous equation, then their difference $Y_1 - Y_2$ is a solution of the associated homogeneous equation.
\end{block}

Inverted, it tells us how to construct new solutions of the inhomogeneous equation from a previously known one: add solutions of the associated homogeneous equation.

\end{frame}

\begin{frame}

\frametitle{The general solution of the inhomogeneous equation}
\label{thegeneralsolutionoftheinhomogeneousequation}

Let us suppose that by some devious method we have constructed a single solution of the inhomogeneous equation, say $ Y $, so that $ aY'' + bY' + cY = g(t) $. Let also $ y_1 $ and $ y_2 $ be a fundamental set of solutions of the associated homogeneous equation.

\begin{block}{General solution of the inhomogeneous equation}
Every function satisfying $ay'' + by' + cy = g(t)$ is of the form
\[ Y + c_1y_1 + c_2 y_2 \]
for some numbers $c_1$ and $c_2$.
\end{block}

Thus these solutions also form a plane: a plane passing through the nonzero point $ Y $.

\end{frame}

\begin{frame}

\frametitle{Construction of particular solutions}
\label{constructionofparticularsolutions}

As usual, this existence theorem doesn't tell us anything about how to construct $ Y $, called a \emph{particular solution} of the equation. We know from the general existence theorem for second-order initial value problems that each has a solution. Methods for finding $ Y $ vary and depend very much on the form of the inhomogeneous term $ g(t) $. We will investigate two such methods: the first of these is \emph{undetermined coefficients}.

\end{frame}

\section{Conclusion}
\label{conclusion}

\begin{frame}

\frametitle{Conclusion: examples good, table bad}
\label{conclusion:examplesgoodtablebad}

By clever guessing with sufficient algebraic stamina, it's possible to build $ Y $ in a reasonably systematic and efficient way when $ g(t) $ is, roughly speaking, built from polynomials, exponential functions, and sine and cosine. But I have one warning to give you.
\pause

It is, very probably, a big mistake to try to memorize the table given in the text. I've watched people try and fail for years. It is a much better idea to learn the mental yoga of how the method works, by working through and thinking about lots of examples.

\end{frame}

\mode<all>
\input{mmd-beamer-footer}

\end{document}\mode*

