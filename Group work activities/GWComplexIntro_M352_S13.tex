\documentclass[11pt]{exam}
% \usepackage{pslatex}
\usepackage{xparse}
\usepackage{graphicx}
\DeclareGraphicsExtensions{.jpg, .png}
\usepackage{amsmath}
\usepackage{amsfonts}
\usepackage{enumerate}
\usepackage{siunitx}
\firstpageheader{}{}{}
\runningheader{\textbf{Spring 2013}}
 {}
 {\textbf{Math 352}}
 %{\emph{Page \thepage~of \numpages}}
\runningheadrule
\setlength{\parskip}{1ex}
\setlength{\parindent}{0pt}
\pagestyle{head}
%\newcommand{\N}{\mathbb{N}}
%\newcommand{\Z}{\mathbb{Z}}
%\newcommand{\R}{\mathbb{R}}
%\newcommand{\dwrspace}[1]{\vspace*{\stretch{#1}}}
\NewDocumentCommand\N{}{\mathbf{N}}
\NewDocumentCommand\R{}{\mathbf{R}}
\NewDocumentCommand\Z{}{\mathbf{Z}}
\NewDocumentCommand\Q{}{\mathbf{Q}}
\NewDocumentCommand\dwrspace{m}{\vspace*{\stretch{#1}}}
\begin{document}
\noindent
\textbf{{\large Mathematics 352 \\ Introduction to complex numbers}}
% \hfill Name: \underline{\hspace{0.5in}Answers\hspace{2in}}

\noindent
March 15, 2013 \hfill Name: \underline{\hspace{3in}} 

\noindent
Due: March 18, 2013

\noindent
\begin{figure}[h]
\centering
\begin{minipage}[b]{0.85\linewidth}
\textbf{Introduction.} In this worksheet, you will investigate some problems in complex algebra arising from second-order ODEs.
\end{minipage}
\end{figure}
\begin{questions}  

\question Find the roots of the characteristic equation for the differential equation $y'' + y' + y = 0$. Note that the discriminant is negative, so they are complex. Call them $r_1$ and $r_2$.

\dwrspace{1}

\question Evaluate $(r_1 + r_2)/2$ and $(r_1 - r_2)/2i$. What do you notice?

\dwrspace{1}

\question What complex exponential functions solve the differential equation? Use the roots you found and ape the exponential trick from before.

\dwrspace{1}

\question Write these functions using Euler's formula. Call them $y_1$ and $y_2$ in their new, Eulerified form.

\dwrspace{1}

\question Write down $(y_1 + y_2)/2$ and $(y_1 - y_2)/2i$ and simplify. What do you notice?

\dwrspace{1}

\begin{center}
    {\large \textsc{Have a great weekend!}}
\end{center}

\end{questions}

\end{document}