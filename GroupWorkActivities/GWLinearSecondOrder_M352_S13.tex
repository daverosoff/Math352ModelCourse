\documentclass[11pt]{exam}
% \usepackage{pslatex}
\usepackage{xparse}
\usepackage{graphicx}
\DeclareGraphicsExtensions{.jpg, .png}
\usepackage{amsmath}
\usepackage{amsfonts}
\usepackage{enumerate}
\usepackage{siunitx}
\firstpageheader{}{}{}
\runningheader{\textbf{Spring 2013}}
 {}
 {\textbf{Math 352}}
 %{\emph{Page \thepage~of \numpages}}
\runningheadrule
\setlength{\parskip}{1ex}
\setlength{\parindent}{0pt}
\pagestyle{head}
%\newcommand{\N}{\mathbb{N}}
%\newcommand{\Z}{\mathbb{Z}}
%\newcommand{\R}{\mathbb{R}}
%\newcommand{\dwrspace}[1]{\vspace*{\stretch{#1}}}
\NewDocumentCommand\N{}{\mathbf{N}}
\NewDocumentCommand\R{}{\mathbf{R}}
\NewDocumentCommand\Z{}{\mathbf{Z}}
\NewDocumentCommand\Q{}{\mathbf{Q}}
\NewDocumentCommand\dwrspace{m}{\vspace*{\stretch{#1}}}
\begin{document}
\noindent
\textbf{{\large Mathematics 352 \\ Second-order linear equations}}
% \hfill Name: \underline{\hspace{0.5in}Answers\hspace{2in}}

\noindent
March 11, 2013 \hfill Name: \underline{\hspace{3in}} 

\noindent
Due: March 13, 2013

\noindent
\begin{figure}[h]
\centering
\begin{minipage}[b]{0.85\linewidth}
\textbf{Introduction.} In this worksheet, you will investigate the simplest kind of second-order differential equations and initial value problems: linear with constant coefficients. Since there are two integrations that must be performed, the solution of such a general equation should involve \emph{two} arbitrary constants. Therefore we expect to see a \emph{2-dimensional} family of solutions (a mathematical object is 2-dimensional if it takes 2 numbers to describe a point on it).
\end{minipage}
\end{figure}
\begin{questions}  

\question Consider the differential equation
\[
    y'' = y.
\]
Think about the familiar functions until you spot a solution. Compare with the people around you. You should be able to find two really ``different'' solutions. Call them $y_1$ and $y_2$. Remember, when we say that $y_1$ is a \emph{solution} to the differential equation, we mean that substitution of $y_1$ for $y$ makes the equation true.

\dwrspace{1}

\question Check by using the substitution criterion that $c_1 y_1$ and $c_2 y_2$ for arbitrary constants $c_1 $ and $c_2 $ are also solutions to $y'' = y$, for the functions $y_1$ and $y_2$ defined above.

\dwrspace{1}

\newpage

\question Use the substitution criterion to check that $c_1 y_1 + c_2 y_2$ is a solution to $y'' = y$.

\dwrspace{1}

\question Initial value problems look a little different for second-order equations. Since there is a 2-dimensional family of solutions, we need 2 initial conditions to pick one out. These most frequently take the form
\[
    y(0) = y_0, \quad y'(0) = y'_0.
\]
Find a solution to the initial value problem $y'' = y$, $y(0) = 1$, $y'(0) = 0$ by plugging in $c_1 y_1 + c_2y_2$ for $y$ and using the initial conditions to find $c_1 $ and $c_2$. How many solutions of this form are there?

\dwrspace{1}

\question Let $r \ne 0$ and consider the function $y = e^{rt}$. Find $y'$ and $y''$, and write $ay'' + by' + cy$ in terms of $y$ (i.e., find an equivalent expression with no primes). You should be able to factor $y$ out of the expression you find. 

\dwrspace{1}

\question Now consider the differential equation $ay'' + by' + cy = 0$. Assume that this equation has a solution of the form $e^{rt}$, and use the result of the previous part to express $r$ in terms of $a$, $b$, $c$.

\end{questions}

\end{document}