\documentclass[answers,12pt]{exam}
% \usepackage{pslatex}
\usepackage{xparse}
\usepackage{graphicx}
\DeclareGraphicsExtensions{.jpg, .png}
\usepackage{amsmath}
%\usepackage{amsfonts}
\usepackage{fourier}
\usepackage{enumerate}
\firstpageheader{}{}{}
\runningheader{\textbf{Spring 2013}}
 {}
 {\textbf{Math 352}}
 %{\emph{Page \thepage~of \numpages}}
\runningheadrule
\setlength{\parskip}{1ex}
\setlength{\parindent}{0pt}
\pagestyle{head}
%\newcommand{\N}{\mathbb{N}}
%\newcommand{\Z}{\mathbb{Z}}
%\newcommand{\R}{\mathbb{R}}
%\newcommand{\dwrspace}[1]{\vspace*{\stretch{#1}}}
\NewDocumentCommand\N{}{\mathbf{N}}
\NewDocumentCommand\R{}{\mathbf{R}}
\NewDocumentCommand\Z{}{\mathbf{Z}}
\NewDocumentCommand\Q{}{\mathbf{Q}}
\NewDocumentCommand\dwrspace{m}{\vspace*{\stretch{#1}}}
\begin{document}
\noindent
\textbf{{\large Mathematics 352 \\ Quiz 8}}
% \hfill Name: \underline{\hspace{0.5in}Answers\hspace{2in}}

\noindent
April 12, 2013; 10 minutes  \hfill Name: \underline{\hspace{3in}} 

\noindent
This quiz is \emph{open-note}, but no books or calculators.

\begin{questions}  

\question Use the method of undetermined coefficients to find the general solution of the differential equation
\[
    y'' - 2y' - 3y = 3e^{2t}.
\]

\begin{solution}
    The characteristic polynomial is $r^2 - 2r - 3 = (r-3)(r+1)$, and so the complementary solution $y_c$ has the form
    \[
        y_c = c_1 e^{3t} + c_2 e^{-t}.
    \]
    Therefore, we can look for a solution $Y$ of the form 
    \[
        Y = Ae^{2t}.
    \]
    Differentiating, we find that $Y' = 2Ae^{2t}$ and $Y'' = 4Ae^{2t}$, and therefore that if $Y$ is a solution to the equation, we have
    \[
        (4Ae^{2t}) - 2(2Ae^{2t}) - 3Ae^{2t} = 3e^{2t},
    \]
    which implies that $A = -1$.

    Therefore, the general solution of the differential equation is $y_c + Y$, which we can write as
    \[
        c_1 y_1 e^{3t} + c_2 y_2 e^{-t} - e^{2t}.
    \]
\end{solution}

\question Use the method of undetermined coefficients to find the general solution of the differential equation
\[
    y'' - 2y' - 3y = -3te^{-t}.
\]

\begin{solution}
    The characteristic polynomial is the same as in the previous problem, so that the complementary solution $y_c$ takes the same form:
    \[
        y_c = c_1 e^{3t} + c_2 e^{-t}.
    \]
    Observe that the right-hand side $g(t)$ is the product of a polynomial of degree one with an etponential. The general yoga of undetermined coefficients tells us that in this case, we should first look for a solution of the form $(At+B)e^{-t}$. 

    However, the same yoga advises us that when \emph{part of the inhomogeneous term appears in the general solution, we must multiply our guess by $t$}. This is because the entire $Be^{-t}$ from above is a solution of the associated homogeneous equation. Therefore we are looking for a solution of the form
    \[
        Y = At^2 e^{-t} + Bt e^{-t}.
    \]
    Differentiating, we find that $Y' = (-At^2 + (2A-B)t + B)e^{-t}$ and $Y'' = (At^2 + (-4A + B)t + (2A - 2B))e^{-t}$. Substituting, we find that
    \[
        -8A t e^{-t} + 2A e^{-t} - 4B e^{-t} = -3te^{-t}.
    \]
    This yields the two algebraic equations
    \begin{align*}
        -8A     &= -3 \\
        2A - 4B &= 0.
    \end{align*}
    Therefore, $A = 3/8$ and $B = 3/16$, and the general solution is
    \[
        c_1 e^{3t} + c_2 e^{-t} + \frac{3}{8} t^2 e^{-t} + \frac{3}{16} t e^{-t}.
    \]
\end{solution}

\end{questions}

\end{document}